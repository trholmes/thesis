% Chapter X

\chapter{The ATLAS Detector} % Chapter title

\label{ch:atlas} % For referencing the chapter elsewhere, use \autoref{ch:name} 

The ATLAS detector circumscribes the LHC ring, enclosing the collision point with a series of particle detecting subsystems, aimed at making as many measurements of the particles leaving the interaction as possible. Its goal is to get a precise measurement of all the stable particles flying from proton-proton collisions at its center, allowing analyzers to fully reconstruct the kinematics of the underlying event.

%----------------------------------------------------------------------------------------

\section{Subdetectors of ATLAS}

The ATLAS Detector consists of many layers of detectors, each of which contributes to the measurements of the position and energy of particles in different ways. This section will describe each of these detetors, in the order in which they are traversed by a particle coming from the collision point. 

%------------------------------------------------

\subsection{The Pixel Detector}

The Pixel detector lies closest to the beam pipe of the \ac{LHC}, and has four layers comprising 92 million read-out channels. There are three standard layers, referred to as L1-L3, and an additional layer added for the 2015 datataking, called the \ac{IBL}. 

\subsubsection{The Original Pixel Detector}



\subsubsection{Addition of the IBL}

In 2015, the \ac{IBL} was lowered into the ATLAS cavern and added to the Pixel Detector. As its name suggests, it was added to improve detection of $B$ mesons, whose non-trivial lifetimes create secondary vertices in ATLAS events, which allow them to be distinguished from other particles with precise track measurement. The \ac{IBL} is closer to the interaction point and has a smaller spacing between sensors, giving it a better chance to see these slightly displaced vertices.

