% Chapter 1

\chapter{Introduction} % Chapter title

\label{ch:introduction} % For referencing the chapter elsewhere, use \autoref{ch:introduction} 

%----------------------------------------------------------------------------------------

In 2010, the \ac{LHC} began colliding protons in its 27 km ring, taking its place as the most powerful in a long line of accelerators aimed at uncovering the fundamental rules that govern particle physics. Its primary goal was to complete the Standard Model of particle physics by discovering the Higgs boson, the last remaining particle that physicists felt sure must exist. With its presence, the Standard Model would be consistent, explaining every observed interaction of known particles, with a complete mathematic framework to describe each feature. However, even with a Higgs boson, the Standard Model contained hints that it might be incomplete, suspicious features that suggested that at a higher energy, there might be something more.

In 2012, the ATLAS and \ac{CMS} Experiments discovered the Higgs boson, leaving the \ac{LHC} physics community without a single primary goal, but rather a host of theories to explore, each extending the Standard Model in a different way. Each theory attempts to solve one of the mysteries left by the Standard Model, providing an explanation for Dark Matter, suggesting a mechanism that could explain Gravity's weakness, or explaining the Higgs boson's mass. For decades, the most popular of these has been Supersymmetry, which proposes a fermionic symmetry and requires a menagerie of new Supersymmetric particles, none of which has yet been observed. 

Supersymmetry simultaneously solves more of the Standard Model's problems than any other, making it appealing to theorists and experimentalists alike. But in order to do this, Supersymmetric particles must appear with masses of approximately 1 \tev, precisely the range of energies the \ac{LHC} is capable of exploring. In 2015, after a three-year shutdown, the \ac{LHC} nearly doubled the energy of its collisions, opening up new territory to be explored by analyzers, and providing data that could either discover or exclude many Supersymmetric models.

The analysis presented in this thesis searches for Supersymmetry, seeking to identify events in which Supersymmetric particles are produced in proton-proton collisions, then decay via a $Z$ boson to a chargeless Supersymmetric particle which escapes ATLAS without detection. A similar ATLAS search, performed with data from the lower-energy collisions 2012, observed a $3\sigma$ excess of events over the expected Standard Model background \cite{SUSY-2014-10}. 

The excess generated a great deal of interest in this channel, and re-investigating it became a top priority when the upgraded \ac{LHC} turned back on in 2015. A preliminary search, performed using the 2015 data only, was released at the end of that year. Again an excess was observed, this time with a significance of $2.2\sigma$ \cite{ATLAS-CONF-2015-082}. 

This thesis describes a search for Supersymmetry performed in this channel using data taken by the ATLAS detector in 2015 and 2016, including an explanation of the theory and motivation behind the search, and a description of the \ac{LHC} and the ATLAS detector. The remaining chapters are laid out as follows: 

\paragraph{Chapter 2} outlines the Standard Model of Particle Physics and the benefits of extending it to include Supersymmetry, then continues on to introduce the specific models used in the search presented in later chapters. It also provides an overview of the process of generating \ac{MC} for use in the ATLAS experiment.

\paragraph{Chapter 3} describes the \ac{LHC} and its operation, including the magnet system, the preaccelerator complex, and some of the phenomenology of collisions at 13 \tev.

\paragraph{Chapter 4} contains descriptions of the many pieces of the ATLAS detector, and how they serve to detect particles coming from \ac{LHC} collisions. ATLAS's magnet and trigger systems are also discussed.

\paragraph{Chapter 5} details the process of reconstruction, the procedue by which the electric signals in the ATLAS detector are interpreted as particles to be used for analysis. 

\paragraph{Chapter 6} presents a neural network designed to improve tracking in the ATLAS Pixel Detector, and describes the benefits of its implementation. 

\paragraph{Chapter 7} lists the main backgrounds for the Supersymmetry search described in this thesis, and provides general ideas of how they can be reduced. 

\paragraph{Chapter 8} outlines how objects are identified and selected for this analysis, referencing many of the working points defined in Chapter 5.  

\paragraph{Chapter 9} explains the analysis's search strategy, defining signal, control, and validation regions, and briefly describing how each contributes to the search.

\paragraph{Chapter 10} describes, for each of the backgrounds described in Chapter 7, how estimates of the Standard Model contributions to the signal region are performed. 

\paragraph{Chapter 11} builds off of Chapter 11, and continues to detail how the uncertainties on each estimate are assessed. 

\paragraph{Chapter 12} shows the results of the analysis, comparing expectations based on background estimates to the observed data. 

\paragraph{Chapter 13} provides interpretations of the results, and explains the statistical procedure used to define exclusions on Supersymmetric models. 

\paragraph{Chapter 14} concludes with a summary of the results, and an outlook for future searches. 
