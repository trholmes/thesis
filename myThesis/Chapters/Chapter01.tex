% Chapter 1

\chapter{Introduction} % Chapter title

\label{ch:introduction} % For referencing the chapter elsewhere, use \autoref{ch:introduction} 

%----------------------------------------------------------------------------------------
The pages that follow detail the author's work on the ATLAS experiment from 2011 through 2016, focusing on an analysis of $13 TeV$ proton-proton collisions at the \ac{LHC} looking for Supersymmetry with the ATLAS Detector. 

\paragraph{Chapter 2} outlines the Standard Model of Particle Physics and the benefits of extending it to include Supersymmetry, then continues on to introduce the specific models used in the search presented in later chapters. It also provides an overview of the process of generating \ac{MC} for use in the ATLAS experiment.

\paragraph{Chapter 3} describes the \ac{LHC} and its operation, including the magnet system, the preaccelerator complex, and some of the phenomenology of collisions at 13 \tev.

\paragraph{Chapter 4} contains descriptions of the many pieces of the ATLAS detector, and how they serve to detect particles coming from \ac{LHC} collisions. ATLAS's magnet and trigger systems are also discussed.

\paragraph{Chapter 5} details the process of reconstruction, the procedue by which the electric signals in the ATLAS detector are interpreted as particles to be used for analysis. 

\paragraph{Chapter 6} presents a neural network designed to improve tracking in the ATLAS Pixel Detector, and describes the benefits of its implementation. 

\paragraph{Chapter 7} lists the main backgrounds for the Supersymmetry search described in this thesis, and provides general ideas of how they can be reduced. 

\paragraph{Chapter 8} outlines how objects are identified and selected for this analysis, referencing many of the working points defined in Chapter 5.  

\paragraph{Chapter 9} explains the analysis's search strategy, defining signal, control, and validation regions, and briefly describing how each contributes to the search.

\paragraph{Chapter 10} describes, for each of the backgrounds described in Chapter 7, how estimates of the Standard Model contributions to the signal region are performed. 

\paragraph{Chapter 11} builds off of Chapter 11, and continues to detail how the uncertainties on each estimate are assessed. 

\paragraph{Chapter 12} shows the results of the analysis, comparing expectations based on background estimates to the observed data. 

\paragraph{Chapter 13} provides interpretations of the results, and explains the statistical procedure used to define exclusions on Supersymmetric models. 

\paragraph{Chapter 14} concludes with a summary of the results, and an outlook for future searches. 
