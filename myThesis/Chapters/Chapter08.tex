% Chapter 8

\chapter{Object Identification and Selection} % Chapter title
\label{ch:objects} 

This section describes the identification and selection of objects in the events of this analysis. Objects are first required to pass \textit{baseline} selections, which are used for \acf{OR} and the calculation of \met, then tighter \textit{signal} selection cuts are applied, which define the objects considered in the final analysis of events. Definitions are presented for electrons, muons, and jets, which are all required in the \ac{SR} of this analysis, as well as photons, which are used in background estimation. This section refers to object definitions, including quality selections, described in \autoref{ch:reconstruction}.

%----------------------------------------------------------------------------------------

\section{Electrons}

Electrons are reconstructed using the algorithm discussed in \autoref{sec:reco_electrons}. All electrons are required to be within $|\eta|<2.47$, to ensure that all tracks are consistently within the tracking capability of the \ac{ID} and that associated clusters are within the high-granularity portion of the calorimeter. Baseline leptons are required to have $\pt>10\gev$ and pass the \texttt{Loose} quality standard. Signal leptons are further required to be isolated and of \texttt{Medium} quality, and must have $\pt>25\gev$. Additional cuts on impact parameter are made for electrons with the goal of identifying only electrons coming from the hard-scatter vertex. These requirements, and all the other requirements made on the electrons can be seen in \autoref{tab:eledef}. 

\begin{table}[ph!]
\begin{center}
    \begin{tabular}{l|c}
      \hline
      Cut            & Value/description \\
      \hline
      \hline
      \multicolumn{2}{c}{Baseline Electron}\\
      \hline
      Acceptance   & $\pt > 10\,\GeV, |\eta^\mathrm{clust}| < 2.47$ \\
      Quality      & \texttt{Loose} \\
      \hline
      \multicolumn{2}{c}{Signal Electron}\\
      \hline
      Acceptance   & $\pt > 25\,\GeV, |\eta^\mathrm{clust}| < 2.47$ \\
      Quality          & \texttt{Medium} \\
      Isolation        & \texttt{GradientLoose} \\
      \multirow{2}{*}{Impact parameter} & $|z_0 \sin\theta|< 0.5$ mm \\
                       & $|d_0/\sigma_{d_0}|< 5$ \\ 
      \hline
      \hline
\end{tabular}
\end{center}
\caption{Summary of the electron selection criteria. $z_0$ gives the track's distance from the hard-scatter vertex projected in the $z$ direction, while $d_0$ gives the this distance projected onto the $x-y$ plane.}              
\label{tab:eledef}
\end{table}

With these signal requirements, the \ac{ATLAS} detector is 80\% efficient at identifying electrons with a \pt of 25 \gev, which rises to 90\% at \pt>60 \gev~\cite{ATLAS-CONF-2014-032}. %Scale factors are applied to correct \ac{MC} to match data efficiencies. These efficiencies are measured as a function of \pt and $\eta$, and include both electron identification efficiencies and trigger efficiencies. 

\section{Muons}

Muons are reconstructed as discussed in \autoref{sec:reco_muons}. Baseline muons are required to have $\pt>10\gev$ and $|\eta|<2.5$, including muons that can be tracked both by the \ac{ID} and the \ac{MS}, and must pass a \texttt{Medium} quality cut. Signal muons are additionally required to have $\pt>25\gev$, and to be isolated. As with the electrons, impact parameter cuts are made to ensure that the muon is consistent with coming from a decay from the event's primary vertex. Additionally, the muon must not be flagged \texttt{isBadMuon}, a designation which identifies muons with inconsistent \ac{ID} and \ac{MS} \pt measurements. The full set of requirements can be seen in \autoref{tab:muondef}.

\begin{table}[ph!]
  \begin{center}
    \begin{tabular}{l|c}
      \hline
      Cut            & Value/description \\
      \hline
      \hline
      \multicolumn{2}{c}{Baseline Muon}\\
      \hline
      Acceptance     & $\pt > 10\,\GeV, |\eta| < 2.5$ \\
      Quality        & \texttt{Medium}    \\
      \hline
      \multicolumn{2}{c}{Signal Muon}\\
      \hline
      Acceptance     & $\pt > 25\,\GeV, |\eta| < 2.5$ \\
      Quality        & \texttt{Medium}    \\
      Isolation        & \texttt{GradientLoose} \\
      Impact parameter & $|z_0 \sin\theta|< 0.5$ mm \\
                       & $|d_0/\sigma_{d_0}|< 3$ \\ 
      isBadMuon        & \texttt{isBadMuon} Flag vetoed \\
      \hline		  
      \hline
    \end{tabular}
  \caption{Summary of the muon selection criteria. The signal selection requirements are applied on top of the baseline selection.}            
    \label{tab:muondef}
  \end{center}
\end{table}

Muons are identified with an efficiency over 98\% \cite{PERF-2015-10}. Including trigger and isolation requirements, this efficiency drops to about 80\% for muons with $\pt>25\gev$ and 90\% for muons with $\pt>200\gev$. This drop due to trigger inefficiency, and is largely the consequence of incomplete $\eta$ coverage of the \acp{RPC}, discussed in \autoref{sec:reco_muons}. %Scalefactors to correct the \ac{MC} identification efficiencies according to data are used.

\section{Jets}

Jets are reconstructed according to \autoref{sec:reco_jets}, with baseline jets using the anti-$k_t$ algorithm with $R=0.4$ using \ac{EM} clusters. These jets are required to have a minimum \pt of 20 \gev~and $|\eta|<2.8$. For signal jets, this \pt requirement is increased to 30 \gev~and the $\eta$ cut is tightened to $|\eta|<2.5$. \ac{JVT} requirements are enforced on jets with \pt<60 \gev~and $|\eta|<2.4$ to reduce the number of jets from pile-up. The full set of requirements can be seen in \autoref{tab:jetsdef}.

\begin{table}[bh!]
\begin{center}
    \begin{tabular}{l|c}
      \hline
      Cut            & Value/description \\
      \hline
      \hline
      \multicolumn{2}{c}{Baseline jet} \\
      \hline
      Collection     & \texttt{AntiKt4EMTopo} \\
      Acceptance     & $\pt > 20\,\GeV$ , $|\eta |<2.8$ \\
      %Quality        &  {\tt very loose}  \\
      \hline
      \multicolumn{2}{c}{Signal jet} \\
      \hline
      Acceptance     & $\pt > 30\,\GeV$ , $|\eta | < 2.5$ \\ 
      \multirow{2}{*}{JVT}      & $|\mathrm{JVT}|>0.59$ for jets with \\
                                & $\pt <60\,\GeV$ and $|\eta | < 2.4$ \\
      \hline
      \multicolumn{2}{c}{Signal $b$-jet} \\
      \hline 
      $b$-tagger Algorithm      & \texttt{MV2c20} \\
      Efficiency                & $77$~\% \\
      Acceptance                & $\pt > 30\,\GeV$ , $|\eta | < 2.5$ \\ 
      \multirow{2}{*}{JVT}      & $|\mathrm{JVT}|>0.59$ for jets with \\
                                & $\pt <60\,\GeV$ and $|\eta | < 2.4$ \\
      \hline
      \hline
\end{tabular}
\end{center}
\caption{Summary of the jet and $b$-jet selection criteria. The signal selection
  requirements are applied on top of the baseline requirements. }
\label{tab:jetsdef}
\end{table}

Though no $b$-jets are required or vetoed in the \ac{SR} of this analysis, some \acp{CR} use $b$-enhanced and $b$-vetoed regions to determine the impact of heavy flavor, and to isolate diboson processes in \acp{VR}. The specifications for $b$-tagging are also described in \autoref{tab:jetsdef}.

\section{Photons}

Photons are used to estimate the \dyjets background in this analysis, and they are reconstructed according to \autoref{sec:reco_photons}. Baseline and signal photons are nearly identical. Each must pass a \texttt{Tight} selection and an isolation cut, and have $\pt > 25\,\GeV$ as well as $|\eta| < 2.37$. Signal photons with $1.37<|\eta|<1.6$ are rejected due to an discontinuity in the calorimeter which results in inferior energy resolutions in this region. The full selection requirements can be seen in \autoref{tab:photondef}.

\begin{table}[ph!]
  \begin{center}
    \begin{tabular}{l|c}
      \hline
      Cut            & Value/description \\
      \hline
      \hline
      \multicolumn{2}{c}{Baseline Photon}\\
      \hline
      Acceptance        & $\pt > 25\,\GeV, |\eta| < 2.37$   \\
      Quality        & \texttt{tight}    \\
      \hline
      \multicolumn{2}{c}{Signal Photon}\\
      \hline
      \multirow{2}{*}{Acceptance}     & $\pt > 25\,\GeV, |\eta| < 2.37$ \\
                     & rejecting $1.37<|\eta|<1.6$\\
      Quality        & \texttt{tight}    \\
      Isolation        & \texttt{FixedCutTight} \\
      \hline      
      \hline
    \end{tabular}
  \caption{Summary of the photon selection criteria.}            
    \label{tab:photondef}
  \end{center}
\end{table}

